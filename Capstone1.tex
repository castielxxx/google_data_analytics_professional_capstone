\documentclass[]{article}
\usepackage{lmodern}
\usepackage{amssymb,amsmath}
\usepackage{ifxetex,ifluatex}
\usepackage{fixltx2e} % provides \textsubscript
\ifnum 0\ifxetex 1\fi\ifluatex 1\fi=0 % if pdftex
  \usepackage[T1]{fontenc}
  \usepackage[utf8]{inputenc}
\else % if luatex or xelatex
  \ifxetex
    \usepackage{mathspec}
  \else
    \usepackage{fontspec}
  \fi
  \defaultfontfeatures{Ligatures=TeX,Scale=MatchLowercase}
\fi
% use upquote if available, for straight quotes in verbatim environments
\IfFileExists{upquote.sty}{\usepackage{upquote}}{}
% use microtype if available
\IfFileExists{microtype.sty}{%
\usepackage{microtype}
\UseMicrotypeSet[protrusion]{basicmath} % disable protrusion for tt fonts
}{}
\usepackage[margin=1in]{geometry}
\usepackage{hyperref}
\hypersetup{unicode=true,
            pdftitle={Google Data Analytics Professional Certificate Capstone},
            pdfauthor={Castiel},
            pdfborder={0 0 0},
            breaklinks=true}
\urlstyle{same}  % don't use monospace font for urls
\usepackage{color}
\usepackage{fancyvrb}
\newcommand{\VerbBar}{|}
\newcommand{\VERB}{\Verb[commandchars=\\\{\}]}
\DefineVerbatimEnvironment{Highlighting}{Verbatim}{commandchars=\\\{\}}
% Add ',fontsize=\small' for more characters per line
\usepackage{framed}
\definecolor{shadecolor}{RGB}{248,248,248}
\newenvironment{Shaded}{\begin{snugshade}}{\end{snugshade}}
\newcommand{\AlertTok}[1]{\textcolor[rgb]{0.94,0.16,0.16}{#1}}
\newcommand{\AnnotationTok}[1]{\textcolor[rgb]{0.56,0.35,0.01}{\textbf{\textit{#1}}}}
\newcommand{\AttributeTok}[1]{\textcolor[rgb]{0.77,0.63,0.00}{#1}}
\newcommand{\BaseNTok}[1]{\textcolor[rgb]{0.00,0.00,0.81}{#1}}
\newcommand{\BuiltInTok}[1]{#1}
\newcommand{\CharTok}[1]{\textcolor[rgb]{0.31,0.60,0.02}{#1}}
\newcommand{\CommentTok}[1]{\textcolor[rgb]{0.56,0.35,0.01}{\textit{#1}}}
\newcommand{\CommentVarTok}[1]{\textcolor[rgb]{0.56,0.35,0.01}{\textbf{\textit{#1}}}}
\newcommand{\ConstantTok}[1]{\textcolor[rgb]{0.00,0.00,0.00}{#1}}
\newcommand{\ControlFlowTok}[1]{\textcolor[rgb]{0.13,0.29,0.53}{\textbf{#1}}}
\newcommand{\DataTypeTok}[1]{\textcolor[rgb]{0.13,0.29,0.53}{#1}}
\newcommand{\DecValTok}[1]{\textcolor[rgb]{0.00,0.00,0.81}{#1}}
\newcommand{\DocumentationTok}[1]{\textcolor[rgb]{0.56,0.35,0.01}{\textbf{\textit{#1}}}}
\newcommand{\ErrorTok}[1]{\textcolor[rgb]{0.64,0.00,0.00}{\textbf{#1}}}
\newcommand{\ExtensionTok}[1]{#1}
\newcommand{\FloatTok}[1]{\textcolor[rgb]{0.00,0.00,0.81}{#1}}
\newcommand{\FunctionTok}[1]{\textcolor[rgb]{0.00,0.00,0.00}{#1}}
\newcommand{\ImportTok}[1]{#1}
\newcommand{\InformationTok}[1]{\textcolor[rgb]{0.56,0.35,0.01}{\textbf{\textit{#1}}}}
\newcommand{\KeywordTok}[1]{\textcolor[rgb]{0.13,0.29,0.53}{\textbf{#1}}}
\newcommand{\NormalTok}[1]{#1}
\newcommand{\OperatorTok}[1]{\textcolor[rgb]{0.81,0.36,0.00}{\textbf{#1}}}
\newcommand{\OtherTok}[1]{\textcolor[rgb]{0.56,0.35,0.01}{#1}}
\newcommand{\PreprocessorTok}[1]{\textcolor[rgb]{0.56,0.35,0.01}{\textit{#1}}}
\newcommand{\RegionMarkerTok}[1]{#1}
\newcommand{\SpecialCharTok}[1]{\textcolor[rgb]{0.00,0.00,0.00}{#1}}
\newcommand{\SpecialStringTok}[1]{\textcolor[rgb]{0.31,0.60,0.02}{#1}}
\newcommand{\StringTok}[1]{\textcolor[rgb]{0.31,0.60,0.02}{#1}}
\newcommand{\VariableTok}[1]{\textcolor[rgb]{0.00,0.00,0.00}{#1}}
\newcommand{\VerbatimStringTok}[1]{\textcolor[rgb]{0.31,0.60,0.02}{#1}}
\newcommand{\WarningTok}[1]{\textcolor[rgb]{0.56,0.35,0.01}{\textbf{\textit{#1}}}}
\usepackage{graphicx,grffile}
\makeatletter
\def\maxwidth{\ifdim\Gin@nat@width>\linewidth\linewidth\else\Gin@nat@width\fi}
\def\maxheight{\ifdim\Gin@nat@height>\textheight\textheight\else\Gin@nat@height\fi}
\makeatother
% Scale images if necessary, so that they will not overflow the page
% margins by default, and it is still possible to overwrite the defaults
% using explicit options in \includegraphics[width, height, ...]{}
\setkeys{Gin}{width=\maxwidth,height=\maxheight,keepaspectratio}
\IfFileExists{parskip.sty}{%
\usepackage{parskip}
}{% else
\setlength{\parindent}{0pt}
\setlength{\parskip}{6pt plus 2pt minus 1pt}
}
\setlength{\emergencystretch}{3em}  % prevent overfull lines
\providecommand{\tightlist}{%
  \setlength{\itemsep}{0pt}\setlength{\parskip}{0pt}}
\setcounter{secnumdepth}{0}
% Redefines (sub)paragraphs to behave more like sections
\ifx\paragraph\undefined\else
\let\oldparagraph\paragraph
\renewcommand{\paragraph}[1]{\oldparagraph{#1}\mbox{}}
\fi
\ifx\subparagraph\undefined\else
\let\oldsubparagraph\subparagraph
\renewcommand{\subparagraph}[1]{\oldsubparagraph{#1}\mbox{}}
\fi

%%% Use protect on footnotes to avoid problems with footnotes in titles
\let\rmarkdownfootnote\footnote%
\def\footnote{\protect\rmarkdownfootnote}

%%% Change title format to be more compact
\usepackage{titling}

% Create subtitle command for use in maketitle
\providecommand{\subtitle}[1]{
  \posttitle{
    \begin{center}\large#1\end{center}
    }
}

\setlength{\droptitle}{-2em}

  \title{Google Data Analytics Professional Certificate Capstone}
    \pretitle{\vspace{\droptitle}\centering\huge}
  \posttitle{\par}
    \author{Castiel}
    \preauthor{\centering\large\emph}
  \postauthor{\par}
      \predate{\centering\large\emph}
  \postdate{\par}
    \date{13/12/2022}


\begin{document}
\maketitle

\hypertarget{setting-up-environment}{%
\section{Setting up environment}\label{setting-up-environment}}

\begin{Shaded}
\begin{Highlighting}[]
\FunctionTok{library}\NormalTok{(tidyverse)  }\CommentTok{\#helps wrangle data}
\end{Highlighting}
\end{Shaded}

\begin{verbatim}
## Registered S3 methods overwritten by 'ggplot2':
##   method         from 
##   [.quosures     rlang
##   c.quosures     rlang
##   print.quosures rlang
\end{verbatim}

\begin{verbatim}
## Registered S3 method overwritten by 'rvest':
##   method            from
##   read_xml.response xml2
\end{verbatim}

\begin{verbatim}
## -- Attaching packages ----------------------------------------------------------------------------- tidyverse 1.2.1 --
\end{verbatim}

\begin{verbatim}
## v ggplot2 3.1.1       v purrr   0.3.2  
## v tibble  2.1.1       v dplyr   0.8.0.1
## v tidyr   0.8.3       v stringr 1.4.0  
## v readr   1.3.1       v forcats 0.4.0
\end{verbatim}

\begin{verbatim}
## -- Conflicts -------------------------------------------------------------------------------- tidyverse_conflicts() --
## x dplyr::filter() masks stats::filter()
## x dplyr::lag()    masks stats::lag()
\end{verbatim}

\begin{Shaded}
\begin{Highlighting}[]
\FunctionTok{library}\NormalTok{(lubridate)  }\CommentTok{\#helps wrangle date attributes}
\end{Highlighting}
\end{Shaded}

\begin{verbatim}
## 
## Attaching package: 'lubridate'
\end{verbatim}

\begin{verbatim}
## The following object is masked from 'package:base':
## 
##     date
\end{verbatim}

\begin{Shaded}
\begin{Highlighting}[]
\FunctionTok{library}\NormalTok{(ggplot2)  }\CommentTok{\#helps visualize data}
\FunctionTok{getwd}\NormalTok{() }\CommentTok{\#displays your working directory}
\end{Highlighting}
\end{Shaded}

\begin{verbatim}
## [1] "/Users/castielchan/google_data_analytics_capstone"
\end{verbatim}

\begin{Shaded}
\begin{Highlighting}[]
\FunctionTok{setwd}\NormalTok{(}\StringTok{"/Users/castielchan/google\_data\_analytics\_capstone"}\NormalTok{) }\CommentTok{\#sets your working directory to simplify calls to data ... make sure to use your OWN username instead of mine ;)}
\end{Highlighting}
\end{Shaded}

getwd() \#displays your working directory
setwd(``/Users/kevinhartman/Desktop/Divvy\_Exercise/csv'') \#sets your
working directory to simplify calls to data \ldots{} make sure to use
your OWN username instead of mine ;)

\#===================== \# STEP 1: COLLECT DATA \#=====================
\# Upload Divvy datasets (csv files) here

\begin{Shaded}
\begin{Highlighting}[]
\NormalTok{q2\_2019 }\OtherTok{\textless{}{-}} \FunctionTok{read\_csv}\NormalTok{(}\StringTok{"Divvy\_Trips\_2019\_Q2.csv"}\NormalTok{)}
\end{Highlighting}
\end{Shaded}

\begin{verbatim}
## Parsed with column specification:
## cols(
##   `01 - Rental Details Rental ID` = col_double(),
##   `01 - Rental Details Local Start Time` = col_datetime(format = ""),
##   `01 - Rental Details Local End Time` = col_datetime(format = ""),
##   `01 - Rental Details Bike ID` = col_double(),
##   `01 - Rental Details Duration In Seconds Uncapped` = col_number(),
##   `03 - Rental Start Station ID` = col_double(),
##   `03 - Rental Start Station Name` = col_character(),
##   `02 - Rental End Station ID` = col_double(),
##   `02 - Rental End Station Name` = col_character(),
##   `User Type` = col_character(),
##   `Member Gender` = col_character(),
##   `05 - Member Details Member Birthday Year` = col_double()
## )
\end{verbatim}

\begin{Shaded}
\begin{Highlighting}[]
\NormalTok{q3\_2019 }\OtherTok{\textless{}{-}} \FunctionTok{read\_csv}\NormalTok{(}\StringTok{"Divvy\_Trips\_2019\_Q3.csv"}\NormalTok{)}
\end{Highlighting}
\end{Shaded}

\begin{verbatim}
## Parsed with column specification:
## cols(
##   trip_id = col_double(),
##   start_time = col_datetime(format = ""),
##   end_time = col_datetime(format = ""),
##   bikeid = col_double(),
##   tripduration = col_number(),
##   from_station_id = col_double(),
##   from_station_name = col_character(),
##   to_station_id = col_double(),
##   to_station_name = col_character(),
##   usertype = col_character(),
##   gender = col_character(),
##   birthyear = col_double()
## )
\end{verbatim}

\begin{Shaded}
\begin{Highlighting}[]
\NormalTok{q4\_2019 }\OtherTok{\textless{}{-}} \FunctionTok{read\_csv}\NormalTok{(}\StringTok{"Divvy\_Trips\_2019\_Q4.csv"}\NormalTok{)}
\end{Highlighting}
\end{Shaded}

\begin{verbatim}
## Parsed with column specification:
## cols(
##   trip_id = col_double(),
##   start_time = col_datetime(format = ""),
##   end_time = col_datetime(format = ""),
##   bikeid = col_double(),
##   tripduration = col_number(),
##   from_station_id = col_double(),
##   from_station_name = col_character(),
##   to_station_id = col_double(),
##   to_station_name = col_character(),
##   usertype = col_character(),
##   gender = col_character(),
##   birthyear = col_double()
## )
\end{verbatim}

\begin{Shaded}
\begin{Highlighting}[]
\NormalTok{q1\_2020 }\OtherTok{\textless{}{-}} \FunctionTok{read\_csv}\NormalTok{(}\StringTok{"Divvy\_Trips\_2020\_Q1.csv"}\NormalTok{)}
\end{Highlighting}
\end{Shaded}

\begin{verbatim}
## Parsed with column specification:
## cols(
##   ride_id = col_character(),
##   rideable_type = col_character(),
##   started_at = col_datetime(format = ""),
##   ended_at = col_datetime(format = ""),
##   start_station_name = col_character(),
##   start_station_id = col_double(),
##   end_station_name = col_character(),
##   end_station_id = col_double(),
##   start_lat = col_double(),
##   start_lng = col_double(),
##   end_lat = col_double(),
##   end_lng = col_double(),
##   member_casual = col_character()
## )
\end{verbatim}

\#==================================================== \# STEP 2:
WRANGLE DATA AND COMBINE INTO A SINGLE FILE
\#==================================================== \# Compare column
names each of the files \# While the names don't have to be in the same
order, they DO need to match perfectly before we can use a command to
join them into one file

\begin{Shaded}
\begin{Highlighting}[]
\FunctionTok{colnames}\NormalTok{(q3\_2019)}
\end{Highlighting}
\end{Shaded}

\begin{verbatim}
##  [1] "trip_id"           "start_time"        "end_time"         
##  [4] "bikeid"            "tripduration"      "from_station_id"  
##  [7] "from_station_name" "to_station_id"     "to_station_name"  
## [10] "usertype"          "gender"            "birthyear"
\end{verbatim}

\begin{Shaded}
\begin{Highlighting}[]
\FunctionTok{colnames}\NormalTok{(q4\_2019)}
\end{Highlighting}
\end{Shaded}

\begin{verbatim}
##  [1] "trip_id"           "start_time"        "end_time"         
##  [4] "bikeid"            "tripduration"      "from_station_id"  
##  [7] "from_station_name" "to_station_id"     "to_station_name"  
## [10] "usertype"          "gender"            "birthyear"
\end{verbatim}

\begin{Shaded}
\begin{Highlighting}[]
\FunctionTok{colnames}\NormalTok{(q2\_2019)}
\end{Highlighting}
\end{Shaded}

\begin{verbatim}
##  [1] "01 - Rental Details Rental ID"                   
##  [2] "01 - Rental Details Local Start Time"            
##  [3] "01 - Rental Details Local End Time"              
##  [4] "01 - Rental Details Bike ID"                     
##  [5] "01 - Rental Details Duration In Seconds Uncapped"
##  [6] "03 - Rental Start Station ID"                    
##  [7] "03 - Rental Start Station Name"                  
##  [8] "02 - Rental End Station ID"                      
##  [9] "02 - Rental End Station Name"                    
## [10] "User Type"                                       
## [11] "Member Gender"                                   
## [12] "05 - Member Details Member Birthday Year"
\end{verbatim}

\begin{Shaded}
\begin{Highlighting}[]
\FunctionTok{colnames}\NormalTok{(q1\_2020)}
\end{Highlighting}
\end{Shaded}

\begin{verbatim}
##  [1] "ride_id"            "rideable_type"      "started_at"        
##  [4] "ended_at"           "start_station_name" "start_station_id"  
##  [7] "end_station_name"   "end_station_id"     "start_lat"         
## [10] "start_lng"          "end_lat"            "end_lng"           
## [13] "member_casual"
\end{verbatim}

\hypertarget{rename-columns-to-make-them-consistent-with-q1_2020-as-this-will-be-the-supposed-going-forward-table-design-for-divvy}{%
\section{Rename columns to make them consistent with q1\_2020 (as this
will be the supposed going-forward table design for
Divvy)}\label{rename-columns-to-make-them-consistent-with-q1_2020-as-this-will-be-the-supposed-going-forward-table-design-for-divvy}}

\begin{Shaded}
\begin{Highlighting}[]
\NormalTok{(q4\_2019 }\OtherTok{\textless{}{-}} \FunctionTok{rename}\NormalTok{(q4\_2019}
\NormalTok{                   ,}\AttributeTok{ride\_id =}\NormalTok{ trip\_id}
\NormalTok{                   ,}\AttributeTok{rideable\_type =}\NormalTok{ bikeid }
\NormalTok{                   ,}\AttributeTok{started\_at =}\NormalTok{ start\_time  }
\NormalTok{                   ,}\AttributeTok{ended\_at =}\NormalTok{ end\_time  }
\NormalTok{                   ,}\AttributeTok{start\_station\_name =}\NormalTok{ from\_station\_name }
\NormalTok{                   ,}\AttributeTok{start\_station\_id =}\NormalTok{ from\_station\_id }
\NormalTok{                   ,}\AttributeTok{end\_station\_name =}\NormalTok{ to\_station\_name }
\NormalTok{                   ,}\AttributeTok{end\_station\_id =}\NormalTok{ to\_station\_id }
\NormalTok{                   ,}\AttributeTok{member\_casual =}\NormalTok{ usertype))}
\end{Highlighting}
\end{Shaded}

\begin{verbatim}
## # A tibble: 704,054 x 12
##    ride_id started_at          ended_at            rideable_type
##      <dbl> <dttm>              <dttm>                      <dbl>
##  1  2.52e7 2019-10-01 00:01:39 2019-10-01 00:17:20          2215
##  2  2.52e7 2019-10-01 00:02:16 2019-10-01 00:06:34          6328
##  3  2.52e7 2019-10-01 00:04:32 2019-10-01 00:18:43          3003
##  4  2.52e7 2019-10-01 00:04:32 2019-10-01 00:43:43          3275
##  5  2.52e7 2019-10-01 00:04:34 2019-10-01 00:35:42          5294
##  6  2.52e7 2019-10-01 00:04:38 2019-10-01 00:10:51          1891
##  7  2.52e7 2019-10-01 00:04:52 2019-10-01 00:22:45          1061
##  8  2.52e7 2019-10-01 00:04:57 2019-10-01 00:29:16          1274
##  9  2.52e7 2019-10-01 00:05:20 2019-10-01 00:29:18          6011
## 10  2.52e7 2019-10-01 00:05:20 2019-10-01 02:23:46          2957
## # ... with 704,044 more rows, and 8 more variables: tripduration <dbl>,
## #   start_station_id <dbl>, start_station_name <chr>,
## #   end_station_id <dbl>, end_station_name <chr>, member_casual <chr>,
## #   gender <chr>, birthyear <dbl>
\end{verbatim}

\begin{Shaded}
\begin{Highlighting}[]
\NormalTok{(q3\_2019 }\OtherTok{\textless{}{-}} \FunctionTok{rename}\NormalTok{(q3\_2019}
\NormalTok{                   ,}\AttributeTok{ride\_id =}\NormalTok{ trip\_id}
\NormalTok{                   ,}\AttributeTok{rideable\_type =}\NormalTok{ bikeid }
\NormalTok{                   ,}\AttributeTok{started\_at =}\NormalTok{ start\_time  }
\NormalTok{                   ,}\AttributeTok{ended\_at =}\NormalTok{ end\_time  }
\NormalTok{                   ,}\AttributeTok{start\_station\_name =}\NormalTok{ from\_station\_name }
\NormalTok{                   ,}\AttributeTok{start\_station\_id =}\NormalTok{ from\_station\_id }
\NormalTok{                   ,}\AttributeTok{end\_station\_name =}\NormalTok{ to\_station\_name }
\NormalTok{                   ,}\AttributeTok{end\_station\_id =}\NormalTok{ to\_station\_id }
\NormalTok{                   ,}\AttributeTok{member\_casual =}\NormalTok{ usertype))}
\end{Highlighting}
\end{Shaded}

\begin{verbatim}
## # A tibble: 1,640,718 x 12
##    ride_id started_at          ended_at            rideable_type
##      <dbl> <dttm>              <dttm>                      <dbl>
##  1  2.35e7 2019-07-01 00:00:27 2019-07-01 00:20:41          3591
##  2  2.35e7 2019-07-01 00:01:16 2019-07-01 00:18:44          5353
##  3  2.35e7 2019-07-01 00:01:48 2019-07-01 00:27:42          6180
##  4  2.35e7 2019-07-01 00:02:07 2019-07-01 00:27:10          5540
##  5  2.35e7 2019-07-01 00:02:13 2019-07-01 00:22:26          6014
##  6  2.35e7 2019-07-01 00:02:21 2019-07-01 00:07:31          4941
##  7  2.35e7 2019-07-01 00:02:24 2019-07-01 00:23:12          3770
##  8  2.35e7 2019-07-01 00:02:26 2019-07-01 00:28:16          5442
##  9  2.35e7 2019-07-01 00:02:34 2019-07-01 00:28:57          2957
## 10  2.35e7 2019-07-01 00:02:45 2019-07-01 00:29:14          6091
## # ... with 1,640,708 more rows, and 8 more variables: tripduration <dbl>,
## #   start_station_id <dbl>, start_station_name <chr>,
## #   end_station_id <dbl>, end_station_name <chr>, member_casual <chr>,
## #   gender <chr>, birthyear <dbl>
\end{verbatim}

\begin{Shaded}
\begin{Highlighting}[]
\NormalTok{(q2\_2019 }\OtherTok{\textless{}{-}} \FunctionTok{rename}\NormalTok{(q2\_2019}
\NormalTok{                   ,}\AttributeTok{ride\_id =} \StringTok{"01 {-} Rental Details Rental ID"}
\NormalTok{                   ,}\AttributeTok{rideable\_type =} \StringTok{"01 {-} Rental Details Bike ID"} 
\NormalTok{                   ,}\AttributeTok{started\_at =} \StringTok{"01 {-} Rental Details Local Start Time"}  
\NormalTok{                   ,}\AttributeTok{ended\_at =} \StringTok{"01 {-} Rental Details Local End Time"}  
\NormalTok{                   ,}\AttributeTok{start\_station\_name =} \StringTok{"03 {-} Rental Start Station Name"} 
\NormalTok{                   ,}\AttributeTok{start\_station\_id =} \StringTok{"03 {-} Rental Start Station ID"}
\NormalTok{                   ,}\AttributeTok{end\_station\_name =} \StringTok{"02 {-} Rental End Station Name"} 
\NormalTok{                   ,}\AttributeTok{end\_station\_id =} \StringTok{"02 {-} Rental End Station ID"}
\NormalTok{                   ,}\AttributeTok{member\_casual =} \StringTok{"User Type"}\NormalTok{))}
\end{Highlighting}
\end{Shaded}

\begin{verbatim}
## # A tibble: 1,108,163 x 12
##    ride_id started_at          ended_at            rideable_type
##      <dbl> <dttm>              <dttm>                      <dbl>
##  1  2.22e7 2019-04-01 00:02:22 2019-04-01 00:09:48          6251
##  2  2.22e7 2019-04-01 00:03:02 2019-04-01 00:20:30          6226
##  3  2.22e7 2019-04-01 00:11:07 2019-04-01 00:15:19          5649
##  4  2.22e7 2019-04-01 00:13:01 2019-04-01 00:18:58          4151
##  5  2.22e7 2019-04-01 00:19:26 2019-04-01 00:36:13          3270
##  6  2.22e7 2019-04-01 00:19:39 2019-04-01 00:23:56          3123
##  7  2.22e7 2019-04-01 00:26:33 2019-04-01 00:35:41          6418
##  8  2.22e7 2019-04-01 00:29:48 2019-04-01 00:36:11          4513
##  9  2.22e7 2019-04-01 00:32:07 2019-04-01 01:07:44          3280
## 10  2.22e7 2019-04-01 00:32:19 2019-04-01 01:07:39          5534
## # ... with 1,108,153 more rows, and 8 more variables: `01 - Rental Details
## #   Duration In Seconds Uncapped` <dbl>, start_station_id <dbl>,
## #   start_station_name <chr>, end_station_id <dbl>,
## #   end_station_name <chr>, member_casual <chr>, `Member Gender` <chr>,
## #   `05 - Member Details Member Birthday Year` <dbl>
\end{verbatim}

\hypertarget{inspect-the-dataframes-and-look-for-incongruencies}{%
\section{Inspect the dataframes and look for
incongruencies}\label{inspect-the-dataframes-and-look-for-incongruencies}}

\begin{Shaded}
\begin{Highlighting}[]
\FunctionTok{str}\NormalTok{(q1\_2020)}
\end{Highlighting}
\end{Shaded}

\begin{verbatim}
## Classes 'spec_tbl_df', 'tbl_df', 'tbl' and 'data.frame': 426887 obs. of  13 variables:
##  $ ride_id           : chr  "EACB19130B0CDA4A" "8FED874C809DC021" "789F3C21E472CA96" "C9A388DAC6ABF313" ...
##  $ rideable_type     : chr  "docked_bike" "docked_bike" "docked_bike" "docked_bike" ...
##  $ started_at        : POSIXct, format: "2020-01-21 20:06:59" "2020-01-30 14:22:39" ...
##  $ ended_at          : POSIXct, format: "2020-01-21 20:14:30" "2020-01-30 14:26:22" ...
##  $ start_station_name: chr  "Western Ave & Leland Ave" "Clark St & Montrose Ave" "Broadway & Belmont Ave" "Clark St & Randolph St" ...
##  $ start_station_id  : num  239 234 296 51 66 212 96 96 212 38 ...
##  $ end_station_name  : chr  "Clark St & Leland Ave" "Southport Ave & Irving Park Rd" "Wilton Ave & Belmont Ave" "Fairbanks Ct & Grand Ave" ...
##  $ end_station_id    : num  326 318 117 24 212 96 212 212 96 100 ...
##  $ start_lat         : num  42 42 41.9 41.9 41.9 ...
##  $ start_lng         : num  -87.7 -87.7 -87.6 -87.6 -87.6 ...
##  $ end_lat           : num  42 42 41.9 41.9 41.9 ...
##  $ end_lng           : num  -87.7 -87.7 -87.7 -87.6 -87.6 ...
##  $ member_casual     : chr  "member" "member" "member" "member" ...
##  - attr(*, "spec")=
##   .. cols(
##   ..   ride_id = col_character(),
##   ..   rideable_type = col_character(),
##   ..   started_at = col_datetime(format = ""),
##   ..   ended_at = col_datetime(format = ""),
##   ..   start_station_name = col_character(),
##   ..   start_station_id = col_double(),
##   ..   end_station_name = col_character(),
##   ..   end_station_id = col_double(),
##   ..   start_lat = col_double(),
##   ..   start_lng = col_double(),
##   ..   end_lat = col_double(),
##   ..   end_lng = col_double(),
##   ..   member_casual = col_character()
##   .. )
\end{verbatim}

\begin{Shaded}
\begin{Highlighting}[]
\FunctionTok{str}\NormalTok{(q4\_2019)}
\end{Highlighting}
\end{Shaded}

\begin{verbatim}
## Classes 'spec_tbl_df', 'tbl_df', 'tbl' and 'data.frame': 704054 obs. of  12 variables:
##  $ ride_id           : num  25223640 25223641 25223642 25223643 25223644 ...
##  $ started_at        : POSIXct, format: "2019-10-01 00:01:39" "2019-10-01 00:02:16" ...
##  $ ended_at          : POSIXct, format: "2019-10-01 00:17:20" "2019-10-01 00:06:34" ...
##  $ rideable_type     : num  2215 6328 3003 3275 5294 ...
##  $ tripduration      : num  940 258 850 2350 1867 ...
##  $ start_station_id  : num  20 19 84 313 210 156 84 156 156 336 ...
##  $ start_station_name: chr  "Sheffield Ave & Kingsbury St" "Throop (Loomis) St & Taylor St" "Milwaukee Ave & Grand Ave" "Lakeview Ave & Fullerton Pkwy" ...
##  $ end_station_id    : num  309 241 199 290 382 226 142 463 463 336 ...
##  $ end_station_name  : chr  "Leavitt St & Armitage Ave" "Morgan St & Polk St" "Wabash Ave & Grand Ave" "Kedzie Ave & Palmer Ct" ...
##  $ member_casual     : chr  "Subscriber" "Subscriber" "Subscriber" "Subscriber" ...
##  $ gender            : chr  "Male" "Male" "Female" "Male" ...
##  $ birthyear         : num  1987 1998 1991 1990 1987 ...
##  - attr(*, "spec")=
##   .. cols(
##   ..   trip_id = col_double(),
##   ..   start_time = col_datetime(format = ""),
##   ..   end_time = col_datetime(format = ""),
##   ..   bikeid = col_double(),
##   ..   tripduration = col_number(),
##   ..   from_station_id = col_double(),
##   ..   from_station_name = col_character(),
##   ..   to_station_id = col_double(),
##   ..   to_station_name = col_character(),
##   ..   usertype = col_character(),
##   ..   gender = col_character(),
##   ..   birthyear = col_double()
##   .. )
\end{verbatim}

\begin{Shaded}
\begin{Highlighting}[]
\FunctionTok{str}\NormalTok{(q3\_2019)}
\end{Highlighting}
\end{Shaded}

\begin{verbatim}
## Classes 'spec_tbl_df', 'tbl_df', 'tbl' and 'data.frame': 1640718 obs. of  12 variables:
##  $ ride_id           : num  23479388 23479389 23479390 23479391 23479392 ...
##  $ started_at        : POSIXct, format: "2019-07-01 00:00:27" "2019-07-01 00:01:16" ...
##  $ ended_at          : POSIXct, format: "2019-07-01 00:20:41" "2019-07-01 00:18:44" ...
##  $ rideable_type     : num  3591 5353 6180 5540 6014 ...
##  $ tripduration      : num  1214 1048 1554 1503 1213 ...
##  $ start_station_id  : num  117 381 313 313 168 300 168 313 43 43 ...
##  $ start_station_name: chr  "Wilton Ave & Belmont Ave" "Western Ave & Monroe St" "Lakeview Ave & Fullerton Pkwy" "Lakeview Ave & Fullerton Pkwy" ...
##  $ end_station_id    : num  497 203 144 144 62 232 62 144 195 195 ...
##  $ end_station_name  : chr  "Kimball Ave & Belmont Ave" "Western Ave & 21st St" "Larrabee St & Webster Ave" "Larrabee St & Webster Ave" ...
##  $ member_casual     : chr  "Subscriber" "Customer" "Customer" "Customer" ...
##  $ gender            : chr  "Male" NA NA NA ...
##  $ birthyear         : num  1992 NA NA NA NA ...
##  - attr(*, "spec")=
##   .. cols(
##   ..   trip_id = col_double(),
##   ..   start_time = col_datetime(format = ""),
##   ..   end_time = col_datetime(format = ""),
##   ..   bikeid = col_double(),
##   ..   tripduration = col_number(),
##   ..   from_station_id = col_double(),
##   ..   from_station_name = col_character(),
##   ..   to_station_id = col_double(),
##   ..   to_station_name = col_character(),
##   ..   usertype = col_character(),
##   ..   gender = col_character(),
##   ..   birthyear = col_double()
##   .. )
\end{verbatim}

\begin{Shaded}
\begin{Highlighting}[]
\FunctionTok{str}\NormalTok{(q2\_2019)}
\end{Highlighting}
\end{Shaded}

\begin{verbatim}
## Classes 'spec_tbl_df', 'tbl_df', 'tbl' and 'data.frame': 1108163 obs. of  12 variables:
##  $ ride_id                                         : num  22178529 22178530 22178531 22178532 22178533 ...
##  $ started_at                                      : POSIXct, format: "2019-04-01 00:02:22" "2019-04-01 00:03:02" ...
##  $ ended_at                                        : POSIXct, format: "2019-04-01 00:09:48" "2019-04-01 00:20:30" ...
##  $ rideable_type                                   : num  6251 6226 5649 4151 3270 ...
##  $ 01 - Rental Details Duration In Seconds Uncapped: num  446 1048 252 357 1007 ...
##  $ start_station_id                                : num  81 317 283 26 202 420 503 260 211 211 ...
##  $ start_station_name                              : chr  "Daley Center Plaza" "Wood St & Taylor St" "LaSalle St & Jackson Blvd" "McClurg Ct & Illinois St" ...
##  $ end_station_id                                  : num  56 59 174 133 129 426 500 499 211 211 ...
##  $ end_station_name                                : chr  "Desplaines St & Kinzie St" "Wabash Ave & Roosevelt Rd" "Canal St & Madison St" "Kingsbury St & Kinzie St" ...
##  $ member_casual                                   : chr  "Subscriber" "Subscriber" "Subscriber" "Subscriber" ...
##  $ Member Gender                                   : chr  "Male" "Female" "Male" "Male" ...
##  $ 05 - Member Details Member Birthday Year        : num  1975 1984 1990 1993 1992 ...
##  - attr(*, "spec")=
##   .. cols(
##   ..   `01 - Rental Details Rental ID` = col_double(),
##   ..   `01 - Rental Details Local Start Time` = col_datetime(format = ""),
##   ..   `01 - Rental Details Local End Time` = col_datetime(format = ""),
##   ..   `01 - Rental Details Bike ID` = col_double(),
##   ..   `01 - Rental Details Duration In Seconds Uncapped` = col_number(),
##   ..   `03 - Rental Start Station ID` = col_double(),
##   ..   `03 - Rental Start Station Name` = col_character(),
##   ..   `02 - Rental End Station ID` = col_double(),
##   ..   `02 - Rental End Station Name` = col_character(),
##   ..   `User Type` = col_character(),
##   ..   `Member Gender` = col_character(),
##   ..   `05 - Member Details Member Birthday Year` = col_double()
##   .. )
\end{verbatim}

\hypertarget{convert-ride_id-and-rideable_type-to-character-so-that-they-can-stack-correctly}{%
\section{Convert ride\_id and rideable\_type to character so that thßey
can stack
correctly}\label{convert-ride_id-and-rideable_type-to-character-so-that-they-can-stack-correctly}}

\begin{Shaded}
\begin{Highlighting}[]
\NormalTok{q4\_2019 }\OtherTok{\textless{}{-}}  \FunctionTok{mutate}\NormalTok{(q4\_2019, }\AttributeTok{ride\_id =} \FunctionTok{as.character}\NormalTok{(ride\_id)}
\NormalTok{                   ,}\AttributeTok{rideable\_type =} \FunctionTok{as.character}\NormalTok{(rideable\_type)) }
\NormalTok{q3\_2019 }\OtherTok{\textless{}{-}}  \FunctionTok{mutate}\NormalTok{(q3\_2019, }\AttributeTok{ride\_id =} \FunctionTok{as.character}\NormalTok{(ride\_id)}
\NormalTok{                   ,}\AttributeTok{rideable\_type =} \FunctionTok{as.character}\NormalTok{(rideable\_type)) }
\NormalTok{q2\_2019 }\OtherTok{\textless{}{-}}  \FunctionTok{mutate}\NormalTok{(q2\_2019, }\AttributeTok{ride\_id =} \FunctionTok{as.character}\NormalTok{(ride\_id)}
\NormalTok{                   ,}\AttributeTok{rideable\_type =} \FunctionTok{as.character}\NormalTok{(rideable\_type)) }
\end{Highlighting}
\end{Shaded}

\hypertarget{stack-individual-quarters-data-frames-into-one-big-data-frame}{%
\section{Stack individual quarter's data frames into one big data
frame}\label{stack-individual-quarters-data-frames-into-one-big-data-frame}}

\begin{Shaded}
\begin{Highlighting}[]
\NormalTok{all\_trips }\OtherTok{\textless{}{-}} \FunctionTok{bind\_rows}\NormalTok{(q2\_2019, q3\_2019, q4\_2019, q1\_2020)}
\end{Highlighting}
\end{Shaded}

\hypertarget{remove-lat-long-birthyear-and-gender-fields-as-this-data-was-dropped-beginning-in-2020}{%
\section{Remove lat, long, birthyear, and gender fields as this data was
dropped beginning in
2020}\label{remove-lat-long-birthyear-and-gender-fields-as-this-data-was-dropped-beginning-in-2020}}

\begin{Shaded}
\begin{Highlighting}[]
\NormalTok{all\_trips }\OtherTok{\textless{}{-}}\NormalTok{ all\_trips }\SpecialCharTok{\%\textgreater{}\%}  
  \FunctionTok{select}\NormalTok{(}\SpecialCharTok{{-}}\FunctionTok{c}\NormalTok{(start\_lat, start\_lng, end\_lat, end\_lng, birthyear, gender, }\StringTok{"01 {-} Rental Details Duration In Seconds Uncapped"}\NormalTok{, }\StringTok{"05 {-} Member Details Member Birthday Year"}\NormalTok{, }\StringTok{"Member Gender"}\NormalTok{, }\StringTok{"tripduration"}\NormalTok{))}
\end{Highlighting}
\end{Shaded}

\#====================================================== \# STEP 3:
CLEAN UP AND ADD DATA TO PREPARE FOR ANALYSIS
\#====================================================== \# Inspect the
new table that has been created

\begin{Shaded}
\begin{Highlighting}[]
\FunctionTok{colnames}\NormalTok{(all\_trips)  }\CommentTok{\#List of column names}
\end{Highlighting}
\end{Shaded}

\begin{verbatim}
## [1] "ride_id"            "started_at"         "ended_at"          
## [4] "rideable_type"      "start_station_id"   "start_station_name"
## [7] "end_station_id"     "end_station_name"   "member_casual"
\end{verbatim}

\begin{Shaded}
\begin{Highlighting}[]
\FunctionTok{nrow}\NormalTok{(all\_trips)  }\CommentTok{\#How many rows are in data frame?}
\end{Highlighting}
\end{Shaded}

\begin{verbatim}
## [1] 3879822
\end{verbatim}

\begin{Shaded}
\begin{Highlighting}[]
\FunctionTok{dim}\NormalTok{(all\_trips)  }\CommentTok{\#Dimensions of the data frame?}
\end{Highlighting}
\end{Shaded}

\begin{verbatim}
## [1] 3879822       9
\end{verbatim}

\begin{Shaded}
\begin{Highlighting}[]
\FunctionTok{head}\NormalTok{(all\_trips)  }\CommentTok{\#See the first 6 rows of data frame.  Also tail(all\_trips)}
\end{Highlighting}
\end{Shaded}

\begin{verbatim}
## # A tibble: 6 x 9
##   ride_id started_at          ended_at            rideable_type
##   <chr>   <dttm>              <dttm>              <chr>        
## 1 221785~ 2019-04-01 00:02:22 2019-04-01 00:09:48 6251         
## 2 221785~ 2019-04-01 00:03:02 2019-04-01 00:20:30 6226         
## 3 221785~ 2019-04-01 00:11:07 2019-04-01 00:15:19 5649         
## 4 221785~ 2019-04-01 00:13:01 2019-04-01 00:18:58 4151         
## 5 221785~ 2019-04-01 00:19:26 2019-04-01 00:36:13 3270         
## 6 221785~ 2019-04-01 00:19:39 2019-04-01 00:23:56 3123         
## # ... with 5 more variables: start_station_id <dbl>,
## #   start_station_name <chr>, end_station_id <dbl>,
## #   end_station_name <chr>, member_casual <chr>
\end{verbatim}

\begin{Shaded}
\begin{Highlighting}[]
\FunctionTok{str}\NormalTok{(all\_trips)  }\CommentTok{\#See list of columns and data types (numeric, character, etc)}
\end{Highlighting}
\end{Shaded}

\begin{verbatim}
## Classes 'spec_tbl_df', 'tbl_df', 'tbl' and 'data.frame': 3879822 obs. of  9 variables:
##  $ ride_id           : chr  "22178529" "22178530" "22178531" "22178532" ...
##  $ started_at        : POSIXct, format: "2019-04-01 00:02:22" "2019-04-01 00:03:02" ...
##  $ ended_at          : POSIXct, format: "2019-04-01 00:09:48" "2019-04-01 00:20:30" ...
##  $ rideable_type     : chr  "6251" "6226" "5649" "4151" ...
##  $ start_station_id  : num  81 317 283 26 202 420 503 260 211 211 ...
##  $ start_station_name: chr  "Daley Center Plaza" "Wood St & Taylor St" "LaSalle St & Jackson Blvd" "McClurg Ct & Illinois St" ...
##  $ end_station_id    : num  56 59 174 133 129 426 500 499 211 211 ...
##  $ end_station_name  : chr  "Desplaines St & Kinzie St" "Wabash Ave & Roosevelt Rd" "Canal St & Madison St" "Kingsbury St & Kinzie St" ...
##  $ member_casual     : chr  "Subscriber" "Subscriber" "Subscriber" "Subscriber" ...
\end{verbatim}

\begin{Shaded}
\begin{Highlighting}[]
\FunctionTok{summary}\NormalTok{(all\_trips)  }\CommentTok{\#Statistical summary of data. Mainly for numerics}
\end{Highlighting}
\end{Shaded}

\begin{verbatim}
##    ride_id            started_at                 
##  Length:3879822     Min.   :2019-04-01 00:02:22  
##  Class :character   1st Qu.:2019-06-23 07:49:09  
##  Mode  :character   Median :2019-08-14 17:43:38  
##                     Mean   :2019-08-26 00:49:59  
##                     3rd Qu.:2019-10-12 12:10:21  
##                     Max.   :2020-03-31 23:51:34  
##                                                  
##     ended_at                   rideable_type      start_station_id
##  Min.   :2019-04-01 00:09:48   Length:3879822     Min.   :  1.0   
##  1st Qu.:2019-06-23 08:20:27   Class :character   1st Qu.: 77.0   
##  Median :2019-08-14 18:02:04   Mode  :character   Median :174.0   
##  Mean   :2019-08-26 01:14:37                      Mean   :202.9   
##  3rd Qu.:2019-10-12 12:36:16                      3rd Qu.:291.0   
##  Max.   :2020-05-19 20:10:34                      Max.   :675.0   
##                                                                   
##  start_station_name end_station_id  end_station_name   member_casual     
##  Length:3879822     Min.   :  1.0   Length:3879822     Length:3879822    
##  Class :character   1st Qu.: 77.0   Class :character   Class :character  
##  Mode  :character   Median :174.0   Mode  :character   Mode  :character  
##                     Mean   :203.8                                        
##                     3rd Qu.:291.0                                        
##                     Max.   :675.0                                        
##                     NA's   :1
\end{verbatim}

\hypertarget{there-are-a-few-problems-we-will-need-to-fix}{%
\section{There are a few problems we will need to
fix:}\label{there-are-a-few-problems-we-will-need-to-fix}}

\hypertarget{in-the-member_casual-column-there-are-two-names-for-members-member-and-subscriber-and-two-names-for-casual-riders-customer-and-casual.-we-will-need-to-consolidate-that-from-four-to-two-labels.}{%
\section{(1) In the ``member\_casual'' column, there are two names for
members (``member'' and ``Subscriber'') and two names for casual riders
(``Customer'' and ``casual''). We will need to consolidate that from
four to two
labels.}\label{in-the-member_casual-column-there-are-two-names-for-members-member-and-subscriber-and-two-names-for-casual-riders-customer-and-casual.-we-will-need-to-consolidate-that-from-four-to-two-labels.}}

\hypertarget{the-data-can-only-be-aggregated-at-the-ride-level-which-is-too-granular.-we-will-want-to-add-some-additional-columns-of-data-such-as-day-month-year-that-provide-additional-opportunities-to-aggregate-the-data.}{%
\section{(2) The data can only be aggregated at the ride-level, which is
too granular. We will want to add some additional columns of data --
such as day, month, year -- that provide additional opportunities to
aggregate the
data.}\label{the-data-can-only-be-aggregated-at-the-ride-level-which-is-too-granular.-we-will-want-to-add-some-additional-columns-of-data-such-as-day-month-year-that-provide-additional-opportunities-to-aggregate-the-data.}}

\hypertarget{we-will-want-to-add-a-calculated-field-for-length-of-ride-since-the-2020q1-data-did-not-have-the-tripduration-column.-we-will-add-ride_length-to-the-entire-dataframe-for-consistency.}{%
\section{(3) We will want to add a calculated field for length of ride
since the 2020Q1 data did not have the ``tripduration'' column. We will
add ``ride\_length'' to the entire dataframe for
consistency.}\label{we-will-want-to-add-a-calculated-field-for-length-of-ride-since-the-2020q1-data-did-not-have-the-tripduration-column.-we-will-add-ride_length-to-the-entire-dataframe-for-consistency.}}

\hypertarget{there-are-some-rides-where-tripduration-shows-up-as-negative-including-several-hundred-rides-where-divvy-took-bikes-out-of-circulation-for-quality-control-reasons.-we-will-want-to-delete-these-rides.}{%
\section{(4) There are some rides where tripduration shows up as
negative, including several hundred rides where Divvy took bikes out of
circulation for Quality Control reasons. We will want to delete these
rides.}\label{there-are-some-rides-where-tripduration-shows-up-as-negative-including-several-hundred-rides-where-divvy-took-bikes-out-of-circulation-for-quality-control-reasons.-we-will-want-to-delete-these-rides.}}

\hypertarget{in-the-member_casual-column-replace-subscriber-with-member-and-customer-with-casual}{%
\section{In the ``member\_casual'' column, replace ``Subscriber'' with
``member'' and ``Customer'' with
``casual''}\label{in-the-member_casual-column-replace-subscriber-with-member-and-customer-with-casual}}

\hypertarget{before-2020-divvy-used-different-labels-for-these-two-types-of-riders-we-will-want-to-make-our-dataframe-consistent-with-their-current-nomenclature}{%
\section{Before 2020, Divvy used different labels for these two types of
riders \ldots{} we will want to make our dataframe consistent with their
current
nomenclature}\label{before-2020-divvy-used-different-labels-for-these-two-types-of-riders-we-will-want-to-make-our-dataframe-consistent-with-their-current-nomenclature}}

\hypertarget{n.b.-level-is-a-special-property-of-a-column-that-is-retained-even-if-a-subset-does-not-contain-any-values-from-a-specific-level}{%
\section{N.B.: ``Level'' is a special property of a column that is
retained even if a subset does not contain any values from a specific
level}\label{n.b.-level-is-a-special-property-of-a-column-that-is-retained-even-if-a-subset-does-not-contain-any-values-from-a-specific-level}}

\hypertarget{begin-by-seeing-how-many-observations-fall-under-each-usertype}{%
\section{Begin by seeing how many observations fall under each
usertype}\label{begin-by-seeing-how-many-observations-fall-under-each-usertype}}

\begin{Shaded}
\begin{Highlighting}[]
\FunctionTok{table}\NormalTok{(all\_trips}\SpecialCharTok{$}\NormalTok{member\_casual)}
\end{Highlighting}
\end{Shaded}

\begin{verbatim}
## 
##     casual   Customer     member Subscriber 
##      48480     857474     378407    2595461
\end{verbatim}

\hypertarget{reassign-to-the-desired-values-we-will-go-with-the-current-2020-labels}{%
\section{Reassign to the desired values (we will go with the current
2020
labels)}\label{reassign-to-the-desired-values-we-will-go-with-the-current-2020-labels}}

\begin{Shaded}
\begin{Highlighting}[]
\NormalTok{all\_trips }\OtherTok{\textless{}{-}}\NormalTok{  all\_trips }\SpecialCharTok{\%\textgreater{}\%} 
  \FunctionTok{mutate}\NormalTok{(}\AttributeTok{member\_casual =} \FunctionTok{recode}\NormalTok{(member\_casual}
\NormalTok{                                ,}\StringTok{"Subscriber"} \OtherTok{=} \StringTok{"member"}
\NormalTok{                                ,}\StringTok{"Customer"} \OtherTok{=} \StringTok{"casual"}\NormalTok{))}
\end{Highlighting}
\end{Shaded}

\hypertarget{check-to-make-sure-the-proper-number-of-observations-were-reassigned}{%
\section{Check to make sure the proper number of observations were
reassigned}\label{check-to-make-sure-the-proper-number-of-observations-were-reassigned}}

\begin{Shaded}
\begin{Highlighting}[]
\FunctionTok{table}\NormalTok{(all\_trips}\SpecialCharTok{$}\NormalTok{member\_casual)}
\end{Highlighting}
\end{Shaded}

\begin{verbatim}
## 
##  casual  member 
##  905954 2973868
\end{verbatim}

\hypertarget{add-columns-that-list-the-date-month-day-and-year-of-each-ride}{%
\section{Add columns that list the date, month, day, and year of each
ride}\label{add-columns-that-list-the-date-month-day-and-year-of-each-ride}}

\hypertarget{this-will-allow-us-to-aggregate-ride-data-for-each-month-day-or-year-before-completing-these-operations-we-could-only-aggregate-at-the-ride-level}{%
\section{This will allow us to aggregate ride data for each month, day,
or year \ldots{} before completing these operations we could only
aggregate at the ride
level}\label{this-will-allow-us-to-aggregate-ride-data-for-each-month-day-or-year-before-completing-these-operations-we-could-only-aggregate-at-the-ride-level}}

\hypertarget{httpswww.statmethods.netinputdates.html-more-on-date-formats-in-r-found-at-that-link}{%
\section{\texorpdfstring{\url{https://www.statmethods.net/input/dates.html}
more on date formats in R found at that
link}{https://www.statmethods.net/input/dates.html more on date formats in R found at that link}}\label{httpswww.statmethods.netinputdates.html-more-on-date-formats-in-r-found-at-that-link}}

\begin{Shaded}
\begin{Highlighting}[]
\NormalTok{all\_trips}\SpecialCharTok{$}\NormalTok{date }\OtherTok{\textless{}{-}} \FunctionTok{as.Date}\NormalTok{(all\_trips}\SpecialCharTok{$}\NormalTok{started\_at) }\CommentTok{\#The default format is yyyy{-}mm{-}dd}
\NormalTok{all\_trips}\SpecialCharTok{$}\NormalTok{month }\OtherTok{\textless{}{-}} \FunctionTok{format}\NormalTok{(}\FunctionTok{as.Date}\NormalTok{(all\_trips}\SpecialCharTok{$}\NormalTok{date), }\StringTok{"\%m"}\NormalTok{)}
\NormalTok{all\_trips}\SpecialCharTok{$}\NormalTok{day }\OtherTok{\textless{}{-}} \FunctionTok{format}\NormalTok{(}\FunctionTok{as.Date}\NormalTok{(all\_trips}\SpecialCharTok{$}\NormalTok{date), }\StringTok{"\%d"}\NormalTok{)}
\NormalTok{all\_trips}\SpecialCharTok{$}\NormalTok{year }\OtherTok{\textless{}{-}} \FunctionTok{format}\NormalTok{(}\FunctionTok{as.Date}\NormalTok{(all\_trips}\SpecialCharTok{$}\NormalTok{date), }\StringTok{"\%Y"}\NormalTok{)}
\NormalTok{all\_trips}\SpecialCharTok{$}\NormalTok{day\_of\_week }\OtherTok{\textless{}{-}} \FunctionTok{format}\NormalTok{(}\FunctionTok{as.Date}\NormalTok{(all\_trips}\SpecialCharTok{$}\NormalTok{date), }\StringTok{"\%A"}\NormalTok{)}
\end{Highlighting}
\end{Shaded}

\hypertarget{add-a-ride_length-calculation-to-all_trips-in-seconds}{%
\section{Add a ``ride\_length'' calculation to all\_trips (in
seconds)}\label{add-a-ride_length-calculation-to-all_trips-in-seconds}}

\hypertarget{httpsstat.ethz.chr-manualr-devellibrarybasehtmldifftime.html}{%
\section{\texorpdfstring{\url{https://stat.ethz.ch/R-manual/R-devel/library/base/html/difftime.html}}{https://stat.ethz.ch/R-manual/R-devel/library/base/html/difftime.html}}\label{httpsstat.ethz.chr-manualr-devellibrarybasehtmldifftime.html}}

\begin{Shaded}
\begin{Highlighting}[]
\NormalTok{all\_trips}\SpecialCharTok{$}\NormalTok{ride\_length }\OtherTok{\textless{}{-}} \FunctionTok{difftime}\NormalTok{(all\_trips}\SpecialCharTok{$}\NormalTok{ended\_at,all\_trips}\SpecialCharTok{$}\NormalTok{started\_at)}
\end{Highlighting}
\end{Shaded}

\hypertarget{inspect-the-structure-of-the-columns}{%
\section{Inspect the structure of the
columns}\label{inspect-the-structure-of-the-columns}}

\begin{Shaded}
\begin{Highlighting}[]
\FunctionTok{str}\NormalTok{(all\_trips)}
\end{Highlighting}
\end{Shaded}

\begin{verbatim}
## Classes 'spec_tbl_df', 'tbl_df', 'tbl' and 'data.frame': 3879822 obs. of  15 variables:
##  $ ride_id           : chr  "22178529" "22178530" "22178531" "22178532" ...
##  $ started_at        : POSIXct, format: "2019-04-01 00:02:22" "2019-04-01 00:03:02" ...
##  $ ended_at          : POSIXct, format: "2019-04-01 00:09:48" "2019-04-01 00:20:30" ...
##  $ rideable_type     : chr  "6251" "6226" "5649" "4151" ...
##  $ start_station_id  : num  81 317 283 26 202 420 503 260 211 211 ...
##  $ start_station_name: chr  "Daley Center Plaza" "Wood St & Taylor St" "LaSalle St & Jackson Blvd" "McClurg Ct & Illinois St" ...
##  $ end_station_id    : num  56 59 174 133 129 426 500 499 211 211 ...
##  $ end_station_name  : chr  "Desplaines St & Kinzie St" "Wabash Ave & Roosevelt Rd" "Canal St & Madison St" "Kingsbury St & Kinzie St" ...
##  $ member_casual     : chr  "member" "member" "member" "member" ...
##  $ date              : Date, format: "2019-04-01" "2019-04-01" ...
##  $ month             : chr  "04" "04" "04" "04" ...
##  $ day               : chr  "01" "01" "01" "01" ...
##  $ year              : chr  "2019" "2019" "2019" "2019" ...
##  $ day_of_week       : chr  "Monday" "Monday" "Monday" "Monday" ...
##  $ ride_length       : 'difftime' num  446 1048 252 357 ...
##   ..- attr(*, "units")= chr "secs"
\end{verbatim}

\hypertarget{convert-ride_length-from-factor-to-numeric-so-we-can-run-calculations-on-the-data}{%
\section{Convert ``ride\_length'' from Factor to numeric so we can run
calculations on the
data}\label{convert-ride_length-from-factor-to-numeric-so-we-can-run-calculations-on-the-data}}

\begin{Shaded}
\begin{Highlighting}[]
\FunctionTok{is.factor}\NormalTok{(all\_trips}\SpecialCharTok{$}\NormalTok{ride\_length)}
\end{Highlighting}
\end{Shaded}

\begin{verbatim}
## [1] FALSE
\end{verbatim}

\begin{Shaded}
\begin{Highlighting}[]
\NormalTok{all\_trips}\SpecialCharTok{$}\NormalTok{ride\_length }\OtherTok{\textless{}{-}} \FunctionTok{as.numeric}\NormalTok{(}\FunctionTok{as.character}\NormalTok{(all\_trips}\SpecialCharTok{$}\NormalTok{ride\_length))}
\FunctionTok{is.numeric}\NormalTok{(all\_trips}\SpecialCharTok{$}\NormalTok{ride\_length)}
\end{Highlighting}
\end{Shaded}

\begin{verbatim}
## [1] TRUE
\end{verbatim}

\hypertarget{remove-bad-data}{%
\section{Remove ``bad'' data}\label{remove-bad-data}}

\hypertarget{the-dataframe-includes-a-few-hundred-entries-when-bikes-were-taken-out-of-docks-and-checked-for-quality-by-divvy-or-ride_length-was-negative}{%
\section{The dataframe includes a few hundred entries when bikes were
taken out of docks and checked for quality by Divvy or ride\_length was
negative}\label{the-dataframe-includes-a-few-hundred-entries-when-bikes-were-taken-out-of-docks-and-checked-for-quality-by-divvy-or-ride_length-was-negative}}

\hypertarget{we-will-create-a-new-version-of-the-dataframe-v2-since-data-is-being-removed}{%
\section{We will create a new version of the dataframe (v2) since data
is being
removed}\label{we-will-create-a-new-version-of-the-dataframe-v2-since-data-is-being-removed}}

\hypertarget{httpswww.datasciencemadesimple.comdelete-or-drop-rows-in-r-with-conditions-2}{%
\section{\texorpdfstring{\url{https://www.datasciencemadesimple.com/delete-or-drop-rows-in-r-with-conditions-2/}}{https://www.datasciencemadesimple.com/delete-or-drop-rows-in-r-with-conditions-2/}}\label{httpswww.datasciencemadesimple.comdelete-or-drop-rows-in-r-with-conditions-2}}

\begin{Shaded}
\begin{Highlighting}[]
\NormalTok{all\_trips\_v2 }\OtherTok{\textless{}{-}}\NormalTok{ all\_trips[}\SpecialCharTok{!}\NormalTok{(all\_trips}\SpecialCharTok{$}\NormalTok{start\_station\_name }\SpecialCharTok{==} \StringTok{"HQ QR"} \SpecialCharTok{|}\NormalTok{ all\_trips}\SpecialCharTok{$}\NormalTok{ride\_length}\SpecialCharTok{\textless{}}\DecValTok{0}\NormalTok{),]}
\end{Highlighting}
\end{Shaded}

\#===================================== \# STEP 4: CONDUCT DESCRIPTIVE
ANALYSIS \#===================================== \# Descriptive analysis
on ride\_length (all figures in seconds)

\begin{Shaded}
\begin{Highlighting}[]
\FunctionTok{mean}\NormalTok{(all\_trips\_v2}\SpecialCharTok{$}\NormalTok{ride\_length) }\CommentTok{\#straight average (total ride length / rides)}
\end{Highlighting}
\end{Shaded}

\begin{verbatim}
## [1] 1479.139
\end{verbatim}

\begin{Shaded}
\begin{Highlighting}[]
\FunctionTok{median}\NormalTok{(all\_trips\_v2}\SpecialCharTok{$}\NormalTok{ride\_length) }\CommentTok{\#midpoint number in the ascending array of ride lengths}
\end{Highlighting}
\end{Shaded}

\begin{verbatim}
## [1] 712
\end{verbatim}

\begin{Shaded}
\begin{Highlighting}[]
\FunctionTok{max}\NormalTok{(all\_trips\_v2}\SpecialCharTok{$}\NormalTok{ride\_length) }\CommentTok{\#longest ride}
\end{Highlighting}
\end{Shaded}

\begin{verbatim}
## [1] 9387024
\end{verbatim}

\begin{Shaded}
\begin{Highlighting}[]
\FunctionTok{min}\NormalTok{(all\_trips\_v2}\SpecialCharTok{$}\NormalTok{ride\_length) }\CommentTok{\#shortest ride}
\end{Highlighting}
\end{Shaded}

\begin{verbatim}
## [1] 1
\end{verbatim}

\hypertarget{you-can-condense-the-four-lines-above-to-one-line-using-summary-on-the-specific-attribute}{%
\section{You can condense the four lines above to one line using
summary() on the specific
attribute}\label{you-can-condense-the-four-lines-above-to-one-line-using-summary-on-the-specific-attribute}}

\begin{Shaded}
\begin{Highlighting}[]
\FunctionTok{summary}\NormalTok{(all\_trips\_v2}\SpecialCharTok{$}\NormalTok{ride\_length)}
\end{Highlighting}
\end{Shaded}

\begin{verbatim}
##    Min. 1st Qu.  Median    Mean 3rd Qu.    Max. 
##       1     412     712    1479    1289 9387024
\end{verbatim}

\hypertarget{compare-members-and-casual-users}{%
\section{Compare members and casual
users}\label{compare-members-and-casual-users}}

\begin{Shaded}
\begin{Highlighting}[]
\FunctionTok{aggregate}\NormalTok{(all\_trips\_v2}\SpecialCharTok{$}\NormalTok{ride\_length }\SpecialCharTok{\textasciitilde{}}\NormalTok{ all\_trips\_v2}\SpecialCharTok{$}\NormalTok{member\_casual, }\AttributeTok{FUN =}\NormalTok{ mean)}
\end{Highlighting}
\end{Shaded}

\begin{verbatim}
##   all_trips_v2$member_casual all_trips_v2$ride_length
## 1                     casual                3552.7502
## 2                     member                 850.0662
\end{verbatim}

\begin{Shaded}
\begin{Highlighting}[]
\FunctionTok{aggregate}\NormalTok{(all\_trips\_v2}\SpecialCharTok{$}\NormalTok{ride\_length }\SpecialCharTok{\textasciitilde{}}\NormalTok{ all\_trips\_v2}\SpecialCharTok{$}\NormalTok{member\_casual, }\AttributeTok{FUN =}\NormalTok{ median)}
\end{Highlighting}
\end{Shaded}

\begin{verbatim}
##   all_trips_v2$member_casual all_trips_v2$ride_length
## 1                     casual                     1546
## 2                     member                      589
\end{verbatim}

\begin{Shaded}
\begin{Highlighting}[]
\FunctionTok{aggregate}\NormalTok{(all\_trips\_v2}\SpecialCharTok{$}\NormalTok{ride\_length }\SpecialCharTok{\textasciitilde{}}\NormalTok{ all\_trips\_v2}\SpecialCharTok{$}\NormalTok{member\_casual, }\AttributeTok{FUN =}\NormalTok{ max)}
\end{Highlighting}
\end{Shaded}

\begin{verbatim}
##   all_trips_v2$member_casual all_trips_v2$ride_length
## 1                     casual                  9387024
## 2                     member                  9056634
\end{verbatim}

\begin{Shaded}
\begin{Highlighting}[]
\FunctionTok{aggregate}\NormalTok{(all\_trips\_v2}\SpecialCharTok{$}\NormalTok{ride\_length }\SpecialCharTok{\textasciitilde{}}\NormalTok{ all\_trips\_v2}\SpecialCharTok{$}\NormalTok{member\_casual, }\AttributeTok{FUN =}\NormalTok{ min)}
\end{Highlighting}
\end{Shaded}

\begin{verbatim}
##   all_trips_v2$member_casual all_trips_v2$ride_length
## 1                     casual                        2
## 2                     member                        1
\end{verbatim}

\hypertarget{see-the-average-ride-time-by-each-day-for-members-vs-casual-users}{%
\section{See the average ride time by each day for members vs casual
users}\label{see-the-average-ride-time-by-each-day-for-members-vs-casual-users}}

\begin{Shaded}
\begin{Highlighting}[]
\FunctionTok{aggregate}\NormalTok{(all\_trips\_v2}\SpecialCharTok{$}\NormalTok{ride\_length }\SpecialCharTok{\textasciitilde{}}\NormalTok{ all\_trips\_v2}\SpecialCharTok{$}\NormalTok{member\_casual }\SpecialCharTok{+}\NormalTok{ all\_trips\_v2}\SpecialCharTok{$}\NormalTok{day\_of\_week, }\AttributeTok{FUN =}\NormalTok{ mean)}
\end{Highlighting}
\end{Shaded}

\begin{verbatim}
##    all_trips_v2$member_casual all_trips_v2$day_of_week
## 1                      casual                   Friday
## 2                      member                   Friday
## 3                      casual                   Monday
## 4                      member                   Monday
## 5                      casual                 Saturday
## 6                      member                 Saturday
## 7                      casual                   Sunday
## 8                      member                   Sunday
## 9                      casual                 Thursday
## 10                     member                 Thursday
## 11                     casual                  Tuesday
## 12                     member                  Tuesday
## 13                     casual                Wednesday
## 14                     member                Wednesday
##    all_trips_v2$ride_length
## 1                 3773.8351
## 2                  824.5305
## 3                 3372.2869
## 4                  842.5726
## 5                 3331.9138
## 6                  968.9337
## 7                 3581.4054
## 8                  919.9746
## 9                 3682.9847
## 10                 823.9278
## 11                3596.3599
## 12                 826.1427
## 13                3718.6619
## 14                 823.9996
\end{verbatim}

\hypertarget{notice-that-the-days-of-the-week-are-out-of-order.-lets-fix-that.}{%
\section{Notice that the days of the week are out of order. Let's fix
that.}\label{notice-that-the-days-of-the-week-are-out-of-order.-lets-fix-that.}}

\begin{Shaded}
\begin{Highlighting}[]
\NormalTok{all\_trips\_v2}\SpecialCharTok{$}\NormalTok{day\_of\_week }\OtherTok{\textless{}{-}} \FunctionTok{ordered}\NormalTok{(all\_trips\_v2}\SpecialCharTok{$}\NormalTok{day\_of\_week, }\AttributeTok{levels=}\FunctionTok{c}\NormalTok{(}\StringTok{"Sunday"}\NormalTok{, }\StringTok{"Monday"}\NormalTok{, }\StringTok{"Tuesday"}\NormalTok{, }\StringTok{"Wednesday"}\NormalTok{, }\StringTok{"Thursday"}\NormalTok{, }\StringTok{"Friday"}\NormalTok{, }\StringTok{"Saturday"}\NormalTok{))}
\end{Highlighting}
\end{Shaded}

\hypertarget{now-lets-run-the-average-ride-time-by-each-day-for-members-vs-casual-users}{%
\section{Now, let's run the average ride time by each day for members vs
casual
users}\label{now-lets-run-the-average-ride-time-by-each-day-for-members-vs-casual-users}}

\begin{Shaded}
\begin{Highlighting}[]
\FunctionTok{aggregate}\NormalTok{(all\_trips\_v2}\SpecialCharTok{$}\NormalTok{ride\_length }\SpecialCharTok{\textasciitilde{}}\NormalTok{ all\_trips\_v2}\SpecialCharTok{$}\NormalTok{member\_casual }\SpecialCharTok{+}\NormalTok{ all\_trips\_v2}\SpecialCharTok{$}\NormalTok{day\_of\_week, }\AttributeTok{FUN =}\NormalTok{ mean)}
\end{Highlighting}
\end{Shaded}

\begin{verbatim}
##    all_trips_v2$member_casual all_trips_v2$day_of_week
## 1                      casual                   Sunday
## 2                      member                   Sunday
## 3                      casual                   Monday
## 4                      member                   Monday
## 5                      casual                  Tuesday
## 6                      member                  Tuesday
## 7                      casual                Wednesday
## 8                      member                Wednesday
## 9                      casual                 Thursday
## 10                     member                 Thursday
## 11                     casual                   Friday
## 12                     member                   Friday
## 13                     casual                 Saturday
## 14                     member                 Saturday
##    all_trips_v2$ride_length
## 1                 3581.4054
## 2                  919.9746
## 3                 3372.2869
## 4                  842.5726
## 5                 3596.3599
## 6                  826.1427
## 7                 3718.6619
## 8                  823.9996
## 9                 3682.9847
## 10                 823.9278
## 11                3773.8351
## 12                 824.5305
## 13                3331.9138
## 14                 968.9337
\end{verbatim}

\hypertarget{analyze-ridership-data-by-type-and-weekday}{%
\section{analyze ridership data by type and
weekday}\label{analyze-ridership-data-by-type-and-weekday}}

\begin{Shaded}
\begin{Highlighting}[]
\NormalTok{all\_trips\_v2 }\SpecialCharTok{\%\textgreater{}\%} 
  \FunctionTok{mutate}\NormalTok{(}\AttributeTok{weekday =} \FunctionTok{wday}\NormalTok{(started\_at, }\AttributeTok{label =} \ConstantTok{TRUE}\NormalTok{)) }\SpecialCharTok{\%\textgreater{}\%}  \CommentTok{\#creates weekday field using wday()}
  \FunctionTok{group\_by}\NormalTok{(member\_casual, weekday) }\SpecialCharTok{\%\textgreater{}\%}  \CommentTok{\#groups by usertype and weekday}
  \FunctionTok{summarise}\NormalTok{(}\AttributeTok{number\_of\_rides =} \FunctionTok{n}\NormalTok{()                           }\CommentTok{\#calculates the number of rides and average duration }
\NormalTok{            ,}\AttributeTok{average\_duration =} \FunctionTok{mean}\NormalTok{(ride\_length)) }\SpecialCharTok{\%\textgreater{}\%}      \CommentTok{\# calculates the average duration}
  \FunctionTok{arrange}\NormalTok{(member\_casual, weekday)                               }\CommentTok{\# sorts}
\end{Highlighting}
\end{Shaded}

\begin{verbatim}
## # A tibble: 14 x 4
## # Groups:   member_casual [2]
##    member_casual weekday number_of_rides average_duration
##    <chr>         <ord>             <int>            <dbl>
##  1 casual        Sun              181293            3581.
##  2 casual        Mon              103296            3372.
##  3 casual        Tue               90510            3596.
##  4 casual        Wed               92457            3719.
##  5 casual        Thu              102679            3683.
##  6 casual        Fri              122404            3774.
##  7 casual        Sat              209543            3332.
##  8 member        Sun              267965             920.
##  9 member        Mon              472196             843.
## 10 member        Tue              508445             826.
## 11 member        Wed              500329             824.
## 12 member        Thu              484177             824.
## 13 member        Fri              452790             825.
## 14 member        Sat              287958             969.
\end{verbatim}

\hypertarget{lets-visualize-the-number-of-rides-by-rider-type}{%
\section{Let's visualize the number of rides by rider
type}\label{lets-visualize-the-number-of-rides-by-rider-type}}

\begin{Shaded}
\begin{Highlighting}[]
\NormalTok{all\_trips\_v2 }\SpecialCharTok{\%\textgreater{}\%} 
  \FunctionTok{mutate}\NormalTok{(}\AttributeTok{weekday =} \FunctionTok{wday}\NormalTok{(started\_at, }\AttributeTok{label =} \ConstantTok{TRUE}\NormalTok{)) }\SpecialCharTok{\%\textgreater{}\%} 
  \FunctionTok{group\_by}\NormalTok{(member\_casual, weekday) }\SpecialCharTok{\%\textgreater{}\%} 
  \FunctionTok{summarise}\NormalTok{(}\AttributeTok{number\_of\_rides =} \FunctionTok{n}\NormalTok{()}
\NormalTok{            ,}\AttributeTok{average\_duration =} \FunctionTok{mean}\NormalTok{(ride\_length)) }\SpecialCharTok{\%\textgreater{}\%} 
  \FunctionTok{arrange}\NormalTok{(member\_casual, weekday)  }\SpecialCharTok{\%\textgreater{}\%} 
  \FunctionTok{ggplot}\NormalTok{(}\FunctionTok{aes}\NormalTok{(}\AttributeTok{x =}\NormalTok{ weekday, }\AttributeTok{y =}\NormalTok{ number\_of\_rides, }\AttributeTok{fill =}\NormalTok{ member\_casual)) }\SpecialCharTok{+}
  \FunctionTok{geom\_col}\NormalTok{(}\AttributeTok{position =} \StringTok{"dodge"}\NormalTok{)}
\end{Highlighting}
\end{Shaded}

\includegraphics{Capstone1_files/figure-latex/unnamed-chunk-25-1.pdf}

\hypertarget{lets-create-a-visualization-for-average-duration}{%
\section{Let's create a visualization for average
duration}\label{lets-create-a-visualization-for-average-duration}}

\begin{Shaded}
\begin{Highlighting}[]
\NormalTok{all\_trips\_v2 }\SpecialCharTok{\%\textgreater{}\%} 
  \FunctionTok{mutate}\NormalTok{(}\AttributeTok{weekday =} \FunctionTok{wday}\NormalTok{(started\_at, }\AttributeTok{label =} \ConstantTok{TRUE}\NormalTok{)) }\SpecialCharTok{\%\textgreater{}\%} 
  \FunctionTok{group\_by}\NormalTok{(member\_casual, weekday) }\SpecialCharTok{\%\textgreater{}\%} 
  \FunctionTok{summarise}\NormalTok{(}\AttributeTok{number\_of\_rides =} \FunctionTok{n}\NormalTok{()}
\NormalTok{            ,}\AttributeTok{average\_duration =} \FunctionTok{mean}\NormalTok{(ride\_length)) }\SpecialCharTok{\%\textgreater{}\%} 
  \FunctionTok{arrange}\NormalTok{(member\_casual, weekday)  }\SpecialCharTok{\%\textgreater{}\%} 
  \FunctionTok{ggplot}\NormalTok{(}\FunctionTok{aes}\NormalTok{(}\AttributeTok{x =}\NormalTok{ weekday, }\AttributeTok{y =}\NormalTok{ average\_duration, }\AttributeTok{fill =}\NormalTok{ member\_casual)) }\SpecialCharTok{+}
  \FunctionTok{geom\_col}\NormalTok{(}\AttributeTok{position =} \StringTok{"dodge"}\NormalTok{)}
\end{Highlighting}
\end{Shaded}

\includegraphics{Capstone1_files/figure-latex/unnamed-chunk-26-1.pdf}

\#================================================= \# STEP 5: EXPORT
SUMMARY FILE FOR FURTHER ANALYSIS
\#================================================= \# Create a csv file
that we will visualize in Excel, Tableau, or my presentation software \#
N.B.: This file location is for a Mac. If you are working on a PC,
change the file location accordingly (most likely
"C:\Users\YOUR\_USERNAME\Desktop...") to export the data. You can read
more here: \url{https://datatofish.com/export-dataframe-to-csv-in-r/}

\begin{Shaded}
\begin{Highlighting}[]
\NormalTok{counts }\OtherTok{\textless{}{-}} \FunctionTok{aggregate}\NormalTok{(all\_trips\_v2}\SpecialCharTok{$}\NormalTok{ride\_length }\SpecialCharTok{\textasciitilde{}}\NormalTok{ all\_trips\_v2}\SpecialCharTok{$}\NormalTok{member\_casual }\SpecialCharTok{+}\NormalTok{ all\_trips\_v2}\SpecialCharTok{$}\NormalTok{day\_of\_week, }\AttributeTok{FUN =}\NormalTok{ mean)}
\FunctionTok{write.csv}\NormalTok{(counts, }\AttributeTok{file =} \StringTok{\textquotesingle{}avg\_ride\_length.csv\textquotesingle{}}\NormalTok{)}
\end{Highlighting}
\end{Shaded}



\end{document}
